\documentclass{article}
\usepackage[utf8]{inputenc}
\usepackage{geometry}

\title{Rapport inlämning C++}
\author{Anna Bergvall och Assar Orpana}
\date{December 2021}

\begin{document}

\maketitle

\section{Övergripande design: beskriv klasser och funktioner, och deras relationer till varandra.}

Syftet med programmet är att koppla ihop två olika containers utan att behöva kopiera några element. Programmets design är mycket enkel - det består av två klassmallar och ett testprogram. Klassmallen concatenation, som finns i filen \verb 'concatenation.h' är gränssnittet för programmet. En concatenation har två datamedlemmar av typen T& som är referenser till de två containers som ska slås ihop. 







Mallfunktionerna \verb 'toString' och \verb 'print' skrevs för att tolkning av testresultat inte skulle kräva utskrifter.



Endast ett testprogram skrevs. Källkoden till detta ses nedan.




\section{En kort användarinstruktion: hur bygger och testar man programmet? Försök att paketera
så mycket som möjligt med regler i makefilen. Det underlättar både under ert arbete och
gör att denna instruktion blir väldigt enkel att skriva.}

Packa upp .zip-filen och öppna med favoriteditorn. Skriv in \verb "make test" i terminalen. Då körs programmet \verb 'test_join_iterator'.

\section{Brister och kommentarer: Finns det något i lösning som du i efterhand anser borde gjorts
annorlunda? Andra kommentarer?}

Vi har valt att inte implementera const-iteratorer på grund av tidsbrist. Strukturen och planeringen av testprogrammet hade kunnat vara mycket bättre. 

\section{Anteckningar}

Container Type Members: size\_type, iterator och const iterator. Men elementet som finns i en container kallas för containerns value\_type. Dessa ska definieras som aliases i containern vi skapar i denna uppgiften.

\\

Speciell typ av iteratorer i join\_iterators?

\end{document}
